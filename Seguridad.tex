\documentclass[spanish,12pt,a4paper]{article}
\usepackage[T1]{fontenc}
\usepackage[utf8]{inputenc}
\usepackage[spanish]{babel}
\usepackage[left=2cm,right=2cm,top=2.5cm,bottom=2.5cm]{geometry}
\usepackage{hyperref}
\usepackage{orcidlink}
\usepackage{multicol}
\usepackage[backend=biber,style=apa]{biblatex}
\addbibresource{referencias.bib}

\begin{document}
	
	%=====================================================================
	%			TÍTULO Y AUTORES FUERA DE COLUMNA
	%=====================================================================
	\begin{flushright}
		{\LARGE \textbf{La Importancia del Cifrado de extremo a extremo en las comunicaciones digitales hoy en día}}\\[1em]
		
		\textbf{Andrade Oscco, Jose Luis}~\orcidlink{0009-0002-6601-1995} \\
		\href{mailto:1009820212@unajma.edu.pe}{1009820212@unajma.edu.pe} \\[0.5em]
		
		\textbf{Lizana Quispe, Miguel Angel}~\orcidlink{0000-0003-2345-6789} \\
		\href{mailto:mlizana@uni.edu.pe}{mlizana@uni.edu.pe} \\[0.5em]
		
		\textbf{Rojas Merino, Alfredo}~\orcidlink{0000-0003-2345-6789} \\
		\href{mailto:arojas@uni.edu.pe}{arojas@uni.edu.pe} \\[0.5em]
		
		Facultad de Ingeniería, Universidad Nacional José María Arguedas, Andahuaylas, Perú \\
		\today
	\end{flushright}
	
	\vspace{2em} 
	
	%=====================================================================
	%			% INICIO DE CONTENIDO EN DOS COLUMNAS
	%=====================================================================
	

	\begin{multicols}{2}
		
		%=====================================================================
		%			% FORMAS DE CITAR  OJO REGISTRAR AL AUTOR EN rreferencias.bib
		%=====================================================================
		
		\section*{\normalsize SUMARIO}
	
		Según \parencite{perez2020}, la investigación cualitativa ofrece una comprensión profunda de los fenómenos sociales.
		
		\section*{\normalsize RESUMEN}
		En el presente revisión de artículo parte de la problemática es que en la actualidad, las comunicaciones digitales se han convertido en una parte fundamental de la vida cotidiana, tanto a nivel personal como profesional. Sin embargo, este crecimiento ha venido acompañado de un aumento en las amenazas a la privacidad y la seguridad de la información de las empresar y a la persona en sí. Frente a este panorama, el cifrado de extremo a extremo (end-to-end encryption, E2EE) ha emergido como una de las soluciones tecnológicas más efectivas para proteger los datos sensibles que se transmiten a través de plataformas digitales. Este trabajo tiene como objetivo realizar una revisión referencial sobre la importancia del cifrado de extremo a extremo en las comunicaciones digitales contemporáneas, abordando sus principios fundamentales, su funcionamiento técnico, sus aplicaciones más comunes y los desafíos asociados a su implementación.
		
		Se examinan distintas fuentes académicas  y técnicas que destacan el papel central del E2EE en la preservación de la confidencialidad y la integridad de la información, especialmente en aplicaciones de mensajería instantánea, correo electrónico y servicios en la nube. Asimismo, se analizan debates actuales sobre su impacto en la seguridad pública, su compatibilidad con marcos legales y las tensiones entre privacidad individual y vigilancia estatal. A través de esta revisión, se evidencia que el cifrado de extremo a extremo no solo es una medida de protección tecnológica, sino también un componente clave en la defensa de los derechos digitales y la libertad de expresión en el entorno digital. Finalmente, se concluye que su fortalecimiento y promoción son fundamentales para garantizar una comunicación segura y confiable en el siglo XXI.
		
		\section*{\normalsize ABSTRACT}
		In the present article review, part of the problem is that, currently, digital communications have become a fundamental part of daily life, both on a personal and professional level. However, this growth has been accompanied by an increase in threats to the privacy and security of information belonging to companies and individuals. In this context, end-to-end encryption (E2EE) has emerged as one of the most effective technological solutions to protect sensitive data transmitted through digital platforms. This work aims to provide a referential review of the importance of end-to-end encryption in contemporary digital communications, addressing its fundamental principles, technical operation, common applications, and the challenges associated with its implementation.
		
		Various academic and technical sources are examined that highlight the central role of E2EE in preserving the confidentiality and integrity of information, especially in instant messaging applications, email, and cloud services. Additionally, current debates about its impact on public security, its compatibility with legal frameworks, and the tensions between individual privacy and state surveillance are analyzed. Through this review, it is evident that end-to-end encryption is not only a technological protection measure but also a key component in defending digital rights and freedom of expression in the digital environment. Finally, it is concluded that strengthening and promoting E2EE is fundamental to ensuring secure and reliable communication in the twenty-first century.
		
		\section*{\normalsize PALABRAS CLAVE}
		Cifrado de extremo a extremo, seguridad digital, privacidad, comunicaciones digitales, protección de datos, criptografía, mensajería segura, derechos digitales.
		\section*{\normalsize KEYWORDS}
		End-to-end encryption, digital security, privacy, digital communications, data protection, cryptography, secure messaging, digital rights.

	\end{multicols}
	
	%=====================================================================
			% FIN DE CONTENIDO EN DOS COLUMNAS
	%=====================================================================
	
	
	\printbibliography
	
\end{document}
