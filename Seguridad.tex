\documentclass[spanish,12pt,a4paper]{article}
\usepackage[T1]{fontenc}
\usepackage[utf8]{inputenc}
\usepackage[spanish]{babel}
%\usepackage{newtxtext,newtxmath}
\usepackage[left=2cm,right=2cm,top=2.5cm,bottom=2.5cm]{geometry}
\usepackage{hyperref}
\usepackage{orcidlink}
\usepackage{multicol}
\usepackage[backend=biber,style=apa]{biblatex}
\addbibresource{referencias.bib}

\begin{document}
	
	%=====================================================================
	%			TÍTULO Y AUTORES FUERA DE COLUMNA
	%=====================================================================
	\begin{flushright}
		{\LARGE \textbf{La Importancia del Cifrado de extremo a extremo en las comunicaciones digitales hoy en día}}\\[1em]
		
		\textbf{Andrade Oscco, Jose Luis}~\orcidlink{0009-0002-6601-1995} \\
		\href{mailto:1009820212@unajma.edu.pe}{1009820212@unajma.edu.pe} \\[0.5em]
		
		\textbf{Lizana Quispe, Miguel Angel}~\orcidlink{0000-0003-2345-6789} \\
		\href{mailto:mlizana@uni.edu.pe}{mlizana@uni.edu.pe} \\[0.5em]
		
		\textbf{Rojas Merino, Alfredo}~\orcidlink{0000-0003-2345-6789} \\
		\href{mailto:arojas@uni.edu.pe}{arojas@uni.edu.pe} \\[0.5em]
		
		Facultad de Ingeniería, Universidad Nacional José María Arguedas, Andahuaylas, Perú \\
		\today
	\end{flushright}
	
	\vspace{2em} 
	
	%=====================================================================
	%			% INICIO DE CONTENIDO EN DOS COLUMNAS
	%=====================================================================
	

	\begin{multicols}{2}
		
		%=====================================================================
		%			% FORMAS DE CITAR  OJO REGISTRAR AL AUTOR EN rreferencias.bib
		%=====================================================================
		
		\section*{\normalsize SUMARIO}
	
		Según \parencite{perez2020}, la investigación cualitativa ofrece una comprensión profunda de los fenómenos sociales.
		
		\section*{\normalsize RESUMEN}
		En el presente revisión de artículo parte de la problemática es que en la actualidad, las comunicaciones digitales se han convertido en una parte fundamental de la vida cotidiana, tanto a nivel personal como profesional. Sin embargo, este crecimiento ha venido acompañado de un aumento en las amenazas a la privacidad y la seguridad de la información de las empresar y a la persona en sí. Frente a este panorama, el cifrado de extremo a extremo (end-to-end encryption, E2EE) ha emergido como una de las soluciones tecnológicas más efectivas para proteger los datos sensibles que se transmiten a través de plataformas digitales. Este trabajo tiene como objetivo realizar una revisión referencial sobre la importancia del cifrado de extremo a extremo en las comunicaciones digitales contemporáneas, abordando sus principios fundamentales, su funcionamiento técnico, sus aplicaciones más comunes y los desafíos asociados a su implementación.
		
		Se examinan distintas fuentes académicas  y técnicas que destacan el papel central del E2EE en la preservación de la confidencialidad y la integridad de la información, especialmente en aplicaciones de mensajería instantánea, correo electrónico y servicios en la nube. Asimismo, se analizan debates actuales sobre su impacto en la seguridad pública, su compatibilidad con marcos legales y las tensiones entre privacidad individual y vigilancia estatal. A través de esta revisión, se evidencia que el cifrado de extremo a extremo no solo es una medida de protección tecnológica, sino también un componente clave en la defensa de los derechos digitales y la libertad de expresión en el entorno digital. Finalmente, se concluye que su fortalecimiento y promoción son fundamentales para garantizar una comunicación segura y confiable en el siglo XXI.
		
		\section*{\normalsize ABSTRACT}
		In the present article review, part of the problem is that, currently, digital communications have become a fundamental part of daily life, both on a personal and professional level. However, this growth has been accompanied by an increase in threats to the privacy and security of information belonging to companies and individuals. In this context, end-to-end encryption (E2EE) has emerged as one of the most effective technological solutions to protect sensitive data transmitted through digital platforms. This work aims to provide a referential review of the importance of end-to-end encryption in contemporary digital communications, addressing its fundamental principles, technical operation, common applications, and the challenges associated with its implementation.
		
		Various academic and technical sources are examined that highlight the central role of E2EE in preserving the confidentiality and integrity of information, especially in instant messaging applications, email, and cloud services. Additionally, current debates about its impact on public security, its compatibility with legal frameworks, and the tensions between individual privacy and state surveillance are analyzed. Through this review, it is evident that end-to-end encryption is not only a technological protection measure but also a key component in defending digital rights and freedom of expression in the digital environment. Finally, it is concluded that strengthening and promoting E2EE is fundamental to ensuring secure and reliable communication in the twenty-first century.
		
		\section*{\normalsize PALABRAS CLAVE}
		Cifrado de extremo a extremo, seguridad digital, privacidad, comunicaciones digitales, protección de datos, criptografía, mensajería segura, derechos digitales.
		\section*{\normalsize KEYWORDS}
		End-to-end encryption, digital security, privacy, digital communications, data protection, cryptography, secure messaging, digital rights.
		\section*{\normalsize INTRODUCCION}
		En la actualidad, las comunicaciones digitales forman parte inseparable de nuestra vida diaria. Desde enviar un mensaje de texto hasta compartir documentos confidenciales por correo electrónico o realizar videollamadas, gran parte de nuestras interacciones personales, laborales y sociales dependen del uso constante de plataformas digitales. Esta transformación, que ha facilitado enormemente la conectividad global, también ha traído consigo un nuevo conjunto de riesgos relacionados con la privacidad, la protección de datos personales y la seguridad de la información que circula en la red.
		
		A medida que los usuarios depositan cada vez más información sensible en entornos digitales, la necesidad de establecer mecanismos de protección sólidos se ha vuelto urgente. La exposición a amenazas como el robo de identidad, el espionaje digital, los ciberataques o el acceso no autorizado a datos privados es una realidad cada vez más común. En este contexto, el cifrado de extremo a extremo, conocido también como E2EE por sus siglas en inglés, se presenta como una de las herramientas más eficaces para resguardar la confidencialidad y la integridad de las comunicaciones digitales.
		
		El cifrado de extremo a extremo garantiza que los mensajes o archivos enviados solo puedan ser leídos por el emisor y el receptor previstos, sin que ningún intermediario, incluidos los proveedores del servicio, pueda acceder a su contenido. Su uso se ha extendido en aplicaciones ampliamente utilizadas, como WhatsApp, Signal o Telegram, y también en entornos más especializados como plataformas de correo seguro y almacenamiento en la nube. Más allá de su funcionamiento técnico, el E2EE tiene implicaciones significativas en términos de derechos digitales, pues protege no solo la información, sino también libertades fundamentales como la privacidad, la libertad de expresión y el derecho a la comunicación segura.
		
		Este trabajo tiene como propósito realizar una revisión referencial que permita comprender en profundidad la relevancia del cifrado de extremo a extremo en el panorama actual de las comunicaciones digitales. Se abordarán sus bases conceptuales y técnicas, sus principales ámbitos de aplicación, y se examinarán tanto los beneficios como las controversias que giran en torno a su uso. A partir del análisis de distintas fuentes académicas y técnicas, se busca reflexionar sobre el papel que desempeña esta tecnología en un mundo donde la información circula a gran velocidad y donde la confianza digital se ha vuelto un bien cada vez más preciado.
		
		\section*{\normalsize OBJETIVO}
			
		Este trabajo tiene como propósito principal realizar una revisión referencial sobre la importancia del cifrado de extremo a extremo en las comunicaciones digitales contemporáneas. Para ello, se propone analizar sus fundamentos técnicos, sus principales aplicaciones prácticas, y su papel en la protección de la privacidad, los datos personales y los derechos digitales. Además, se busca examinar los desafíos legales, éticos y sociales que enfrenta su implementación en un contexto global marcado por la constante tensión entre la seguridad y la vigilancia.
		
		Se pretende, a través de esta revisión, examinar el papel que desempeña el cifrado de extremo a extremo en la protección de las comunicaciones digitales, considerando tanto sus fundamentos técnicos como sus implicaciones en la privacidad y los derechos digitales. Asimismo, se busca identificar las principales aplicaciones que utilizan esta tecnología y analizar las distintas posturas presentes en la literatura académica respecto a sus beneficios, limitaciones y desafíos legales. Mediante esta exploración, se espera aportar una visión crítica y actualizada sobre el impacto del cifrado E2EE en un entorno digital cada vez más expuesto a riesgos de seguridad y vigilancia.
		\section*{\normalsize METODOLOGIA}
		
		a presente revisión se llevó a cabo siguiendo un enfoque cualitativo de tipo documental, basado en la búsqueda, selección y análisis crítico de fuentes académicas y técnicas relacionadas con el cifrado de extremo a extremo (E2EE) en las comunicaciones digitales. El objetivo fue identificar, comparar y evaluar los principales aportes teóricos y prácticos disponibles en la literatura especializada, con énfasis en la seguridad de la información, la privacidad de los usuarios y los desafíos legales asociados a esta tecnología.
		
		\subsection*{\normalsize Búsqueda bibliográfica y criterios de selección }
		La recolección de la información se realizó mediante una búsqueda sistemática en bases de datos académicas reconocidas, como Scopus, IEEE Xplore, ScienceDirect, Google Scholar y Redalyc, utilizando palabras clave como: “end-to-end encryption”, “digital privacy”, “data protection”, “cryptography”, “secure communications” y sus equivalentes en español. Se limitaron los resultados a publicaciones académicas, artículos científicos, informes técnicos y documentos institucionales publicados entre los años 2015 y 2024, para asegurar la actualidad y relevancia de la información.
		
		Los criterios de inclusión contemplaron trabajos que abordaran aspectos técnicos del cifrado E2EE, sus aplicaciones en plataformas digitales, implicaciones legales, debates éticos y consideraciones en torno a los derechos digitales. Se excluyeron artículos puramente comerciales, sin revisión por pares, o que no profundizaran en el cifrado desde una perspectiva crítica o especializada.
		
		\subsection*{\normalsize Fuentes documentales y recuperación de la información }
		Las fuentes seleccionadas provienen mayoritariamente de revistas científicas indexadas, publicaciones de organismos especializados en seguridad informática y documentos académicos relevantes en el área de la criptografía y la protección de datos. La información fue organizada mediante matrices de análisis temático, lo cual permitió clasificar los artículos según sus principales enfoques (técnico, legal, ético, social) y contrastar sus aportes.
		
		\subsection*{\normalsize Evaluación de calidad, fiabilidad y validez }
		Para garantizar la calidad de los documentos revisados, se consideró el factor de impacto de las revistas, la reputación de los autores, el nivel de citación de los artículos y la claridad metodológica en cada fuente. Se priorizaron publicaciones con revisión por pares, metodologías bien fundamentadas y resultados claramente argumentados. Asimismo, se evaluó la fiabilidad de los datos presentados y la validez interna de los argumentos, considerando su coherencia con otras investigaciones del área.
		
		\subsection*{\normalsize Análisis de la variabilidad y consistencia de las fuentes }
		
		Durante el análisis comparativo, se identificaron tanto puntos de convergencia como diferencias en la forma en que los autores abordan el cifrado de extremo a extremo. Esta variabilidad fue examinada en función del contexto de aplicación (por ejemplo, mensajería, correo electrónico, servicios en la nube) y de las perspectivas disciplinarias (tecnología, derecho, ética). A pesar de ciertas discrepancias, los hallazgos muestran una consistencia general en cuanto al valor del cifrado E2EE como mecanismo de protección y su relevancia creciente en la defensa de la privacidad digital.
		
	\end{multicols}
	
	%=====================================================================
			% FIN DE CONTENIDO EN DOS COLUMNAS
	%=====================================================================
	
	
	\printbibliography
	
\end{document}
