\documentclass[spanish,12pt,a4paper]{article}
\usepackage[spanish]{babel}

\usepackage[T1]{fontenc}
\usepackage[utf8]{inputenc}
\usepackage[spanish]{babel}
%\usepackage{newtxtext,newtxmath}
\usepackage[left=2cm,right=2cm,top=2.5cm,bottom=2.5cm]{geometry}
\usepackage{hyperref}
\usepackage{orcidlink}
\usepackage{multicol}
\usepackage[backend=biber,style=apa]{biblatex}
\addbibresource{referencias.bib}
\usepackage{tabularx}

\usepackage{pgfplots}
\usepackage{pgfplotstable}
\usepackage{graphicx}
\usepackage{caption}
\usepackage{float}  % <- esto es importante
\usepackage{geometry}
\usepackage{enumitem}
% graficos mapa
\geometry{margin=1in}
\usepackage{mathptmx} 
\usepackage{hyperref}

\usepackage{tikz}
\usetikzlibrary{shapes.geometric, arrows.meta, positioning}
\tikzset{
	concept/.style={circle, draw, fill=blue!20, text width=3.5cm, align=center, minimum height=1.2cm, font=\small, inner sep=3pt},
	subconcept/.style={rectangle, draw, fill=green!15, text width=3.5cm, align=left, minimum height=1cm, font=\footnotesize, inner sep=3pt},
	arrow/.style={thick,->,>=stealth}
}
\tikzstyle{box} = [rectangle, rounded corners, minimum width=8cm, minimum height=1cm,text centered, draw=black, fill=blue!20]
\tikzstyle{arrow} = [thick,->,>=stealth]

\usepackage{tikz}
\usetikzlibrary{shapes.geometric, arrows.meta, positioning}

\tikzset{
	box/.style={
		rectangle, rounded corners, minimum width=3.5cm, minimum height=1cm,
		text centered, draw=black, fill=blue!20
	},
	smallbox/.style={
		rectangle, rounded corners, minimum width=4cm, minimum height=0.7cm,
		text centered, draw=black, fill=red!20
	},
	arrow/.style={
		thick,->,>=stealth
	}
}


\begin{document}
	
	%=====================================================================
	%			TÍTULO Y AUTORES FUERA DE COLUMNA
	%=====================================================================
	\begin{flushright}
		{\LARGE \textbf{La Importancia del Cifrado de extremo a extremo en las comunicaciones digitales hoy en día}}\\[1em]
		
		$^{1} $\textbf{José Luis Andrade Oscco}~\orcidlink{0009-0002-6601-1995} \\
		Correo electrónico: \href{mailto:1009820212@unajma.edu.pe}{\texttt{1009820212@unajma.edu.pe}} \\[0.5em]
	
		$^{2} $\textbf{Miguel Ángel Lizana Quispe}~\orcidlink{0000-0003-2345-6789} \\
		Correo electrónico: \href{mailto:mlizana@uni.edu.pe}{\texttt{mlizana@uni.edu.pe}} \\[0.5em]
		
		$^{3} $\textbf{Alfredo Rojas Merino}~\orcidlink{0000-0003-2345-6789} \\
		Correo electrónico: \href{mailto:arojas@uni.edu.pe}{\texttt{arojas@uni.edu.pe}} \\[0.5em]
		
		
		Facultad de Ingeniería, Universidad Nacional José María Arguedas, Andahuaylas, Perú \\
		\today
	\end{flushright}
	
	\vspace{2em} 
	
	%=====================================================================
	%			% INICIO DE CONTENIDO EN DOS COLUMNAS
	%=====================================================================
	

	\begin{multicols}{2}
		
		%=====================================================================
		%			% FORMAS DE CITAR  OJO REGISTRAR AL AUTOR EN rreferencias.bib
		%=====================================================================
		
		\section*{\normalsize SUMARIO}
	
		Según \parencite{perez2020}, la investigación cualitativa ofrece una comprensión profunda de los fenómenos sociales.
		
		\section*{\normalsize RESUMEN}
		En el presente revisión de artículo parte de la problemática es que en la actualidad, las comunicaciones digitales se han convertido en una parte fundamental de la vida cotidiana, tanto a nivel personal como profesional. Sin embargo, este crecimiento ha venido acompañado de un aumento en las amenazas a la privacidad y la seguridad de la información de las empresar y a la persona en sí. Frente a este panorama, el cifrado de extremo a extremo (end-to-end encryption, E2EE) ha emergido como una de las soluciones tecnológicas más efectivas para proteger los datos sensibles que se transmiten a través de plataformas digitales. Este trabajo tiene como objetivo realizar una revisión referencial sobre la importancia del cifrado de extremo a extremo en las comunicaciones digitales contemporáneas, abordando sus principios fundamentales, su funcionamiento técnico, sus aplicaciones más comunes y los desafíos asociados a su implementación.
		
		Se examinan distintas fuentes académicas  y técnicas que destacan el papel central del E2EE en la preservación de la confidencialidad y la integridad de la información, especialmente en aplicaciones de mensajería instantánea, correo electrónico y servicios en la nube. Asimismo, se analizan debates actuales sobre su impacto en la seguridad pública, su compatibilidad con marcos legales y las tensiones entre privacidad individual y vigilancia estatal. A través de esta revisión, se evidencia que el cifrado de extremo a extremo no solo es una medida de protección tecnológica, sino también un componente clave en la defensa de los derechos digitales y la libertad de expresión en el entorno digital. Finalmente, se concluye que su fortalecimiento y promoción son fundamentales para garantizar una comunicación segura y confiable en el siglo XXI.
		
		\section*{\normalsize ABSTRACT}
		In the present article review, part of the problem is that, currently, digital communications have become a fundamental part of daily life, both on a personal and professional level. However, this growth has been accompanied by an increase in threats to the privacy and security of information belonging to companies and individuals. In this context, end-to-end encryption (E2EE) has emerged as one of the most effective technological solutions to protect sensitive data transmitted through digital platforms. This work aims to provide a referential review of the importance of end-to-end encryption in contemporary digital communications, addressing its fundamental principles, technical operation, common applications, and the challenges associated with its implementation.
		
		Various academic and technical sources are examined that highlight the central role of E2EE in preserving the confidentiality and integrity of information, especially in instant messaging applications, email, and cloud services. Additionally, current debates about its impact on public security, its compatibility with legal frameworks, and the tensions between individual privacy and state surveillance are analyzed. Through this review, it is evident that end-to-end encryption is not only a technological protection measure but also a key component in defending digital rights and freedom of expression in the digital environment. Finally, it is concluded that strengthening and promoting E2EE is fundamental to ensuring secure and reliable communication in the twenty-first century.
		
		\section*{\normalsize PALABRAS CLAVE}
		Cifrado de extremo a extremo, seguridad digital, privacidad, comunicaciones digitales, protección de datos, criptografía, mensajería segura, derechos digitales.
		\section*{\normalsize KEYWORDS}
		End-to-end encryption, digital security, privacy, digital communications, data protection, cryptography, secure messaging, digital rights.
		\section*{\normalsize INTRODUCCION}
		En la actualidad, las comunicaciones digitales forman parte inseparable de nuestra vida diaria. Desde enviar un mensaje de texto hasta compartir documentos confidenciales por correo electrónico o realizar videollamadas, gran parte de nuestras interacciones personales, laborales y sociales dependen del uso constante de plataformas digitales. Esta transformación, que ha facilitado enormemente la conectividad global, también ha traído consigo un nuevo conjunto de riesgos relacionados con la privacidad, la protección de datos personales y la seguridad de la información que circula en la red.
		
		A medida que los usuarios depositan cada vez más información sensible en entornos digitales, la necesidad de establecer mecanismos de protección sólidos se ha vuelto urgente. La exposición a amenazas como el robo de identidad, el espionaje digital, los ciberataques o el acceso no autorizado a datos privados es una realidad cada vez más común. En este contexto, el cifrado de extremo a extremo, conocido también como E2EE por sus siglas en inglés, se presenta como una de las herramientas más eficaces para resguardar la confidencialidad y la integridad de las comunicaciones digitales.
		
		El cifrado de extremo a extremo garantiza que los mensajes o archivos enviados solo puedan ser leídos por el emisor y el receptor previstos, sin que ningún intermediario, incluidos los proveedores del servicio, pueda acceder a su contenido. Su uso se ha extendido en aplicaciones ampliamente utilizadas, como WhatsApp, Signal o Telegram, y también en entornos más especializados como plataformas de correo seguro y almacenamiento en la nube. Más allá de su funcionamiento técnico, el E2EE tiene implicaciones significativas en términos de derechos digitales, pues protege no solo la información, sino también libertades fundamentales como la privacidad, la libertad de expresión y el derecho a la comunicación segura.
		
		Este trabajo tiene como propósito realizar una revisión referencial que permita comprender en profundidad la relevancia del cifrado de extremo a extremo en el panorama actual de las comunicaciones digitales. Se abordarán sus bases conceptuales y técnicas, sus principales ámbitos de aplicación, y se examinarán tanto los beneficios como las controversias que giran en torno a su uso. A partir del análisis de distintas fuentes académicas y técnicas, se busca reflexionar sobre el papel que desempeña esta tecnología en un mundo donde la información circula a gran velocidad y donde la confianza digital se ha vuelto un bien cada vez más preciado.
		
		\section*{\normalsize OBJETIVO}
			
		Este trabajo tiene como propósito principal realizar una revisión referencial sobre la importancia del cifrado de extremo a extremo en las comunicaciones digitales contemporáneas. Para ello, se propone analizar sus fundamentos técnicos, sus principales aplicaciones prácticas, y su papel en la protección de la privacidad, los datos personales y los derechos digitales. Además, se busca examinar los desafíos legales, éticos y sociales que enfrenta su implementación en un contexto global marcado por la constante tensión entre la seguridad y la vigilancia.
		
		Se pretende, a través de esta revisión, examinar el papel que desempeña el cifrado de extremo a extremo en la protección de las comunicaciones digitales, considerando tanto sus fundamentos técnicos como sus implicaciones en la privacidad y los derechos digitales. Asimismo, se busca identificar las principales aplicaciones que utilizan esta tecnología y analizar las distintas posturas presentes en la literatura académica respecto a sus beneficios, limitaciones y desafíos legales. Mediante esta exploración, se espera aportar una visión crítica y actualizada sobre el impacto del cifrado E2EE en un entorno digital cada vez más expuesto a riesgos de seguridad y vigilancia.
		%=====================================================================
		%			% METODOLOGIA
		%=====================================================================
		\section*{\normalsize METODOLOGIA}
		
		a presente revisión se llevó a cabo siguiendo un enfoque cualitativo de tipo documental, basado en la búsqueda, selección y análisis crítico de fuentes académicas y técnicas relacionadas con el cifrado de extremo a extremo (E2EE) en las comunicaciones digitales. El objetivo fue identificar, comparar y evaluar los principales aportes teóricos y prácticos disponibles en la literatura especializada, con énfasis en la seguridad de la información, la privacidad de los usuarios y los desafíos legales asociados a esta tecnología.
		
		\subsection*{\normalsize Búsqueda bibliográfica y criterios de selección }
		La recolección de la información se realizó mediante una búsqueda sistemática en bases de datos académicas reconocidas, como Scopus, IEEE Xplore,arXiv entre otros , utilizando palabras clave como: “end-to-end encryption”, “digital privacy”, “data protection”, “cryptography”, “secure communications” y sus equivalentes en español. \textbf{Cuadro}~\ref{tab:busqueda}.
		
		Los criterios de inclusión contemplaron trabajos que abordaran aspectos técnicos del cifrado E2EE, sus aplicaciones en plataformas digitales, implicaciones legales, debates éticos y consideraciones en torno a los derechos digitales. Se excluyeron artículos puramente comerciales, sin revisión por pares, o que no profundizaran en el cifrado desde una perspectiva crítica o especializada.
		
		El proceso seguido para la identificación, selección y exclusión de artículos se resume en el flujograma presentado en la \textbf{Figura}~\ref{fig:flujograma}.
		
		
		
		\subsection*{\normalsize Fuentes documentales y recuperación de la información }
		Las fuentes seleccionadas provienen mayoritariamente de revistas científicas indexadas, publicaciones de organismos especializados en seguridad informática y documentos académicos relevantes en el área de la criptografía y la protección de datos. La información fue organizada mediante matrices de análisis temático, lo cual permitió clasificar los artículos según sus principales enfoques (técnico, legal, ético, social) y contrastar sus aportes.
		\subsection*{\normalsize Evaluación de calidad, fiabilidad y validez }
		Para asegurar la calidad de los documentos revisados, se tomó en cuenta diversos aspectos que permiten garantizar la confiabilidad y pertinencia de la información analizada. En primer lugar, se consideró el prestigio de las revistas donde fueron publicados los artículos, evaluando factores como el factor de impacto y el reconocimiento dentro de la comunidad académica. Además, se revisó la reputación de los autores, poniendo especial atención en su trayectoria y contribuciones previas al área de estudio. Otro aspecto fundamental fue la cantidad de veces que los artículos han sido citados, lo que da una idea de su relevancia y aceptación dentro del campo. También se examinó con detenimiento la claridad y solidez de las metodologías empleadas en cada fuente, priorizando aquellas que presentan procedimientos bien fundamentados y resultados transparentes. Finalmente, se evaluó la coherencia interna de los argumentos presentados, contrastando la información con otras investigaciones para asegurar que no existieran contradicciones o datos poco fiables. De esta manera, se buscó construir una base sólida de conocimiento que permita realizar un análisis riguroso y con fundamentos confiables, evitando así incluir información sesgada o poco pertinente.
		
	\end{multicols}
	
	
	%=====================================================================
	% FIN DE CONTENIDO EN DOS COLUMNAS
	%=====================================================================
			
	\begin{figure}[ht]
		\caption{ Flujograma de selección de artículos}
		\vspace{0.5cm}
		\label{fig:flujograma}
		\centering
		\begin{tikzpicture}[
			node distance=1.1cm and 0.9cm,
			every node/.style={draw, rounded corners, align=center, font=\scriptsize, text width=3.5cm, inner sep=4pt},
			arrow/.style={-{Stealth}, thick}
			]
			
			% Fuentes
			\node (pubmed) {\textit{IEEexplore.}\\n = 15};
			\node[right=of pubmed] (science) {arXiv.orgt\\n = 5};
			\node[right=of science] (scopus) {Scopus\\n = 3};
			
			% Nodo central de total
			\node[below=of science] (total) {Total de artículos\\n = 23};
			
			% Revisión
			\node[below=of total] (revision) {Revisión de \textit{título} y \textit{resumen}};
			
			% Excluidos
			\node[right=3.5cm of revision] (excluidos) {Artículos excluidos (\textbf{duplicados},\\ \textbf{sin acceso a texto completo},\\ \textbf{no relevantes})\\n = 15};
			
			% Incluidos
			\node[below=of revision] (incluidos) {Artículos incluidos en la\\ revisión de tema \textit{general}\\n = 8};
			
			% Flechas
			\draw[arrow] (pubmed) -- (total);
			\draw[arrow] (science) -- (total);
			\draw[arrow] (scopus) -- (total);
			\draw[arrow] (total) -- (revision);
			\draw[arrow] (revision) -- (incluidos);
			\draw[arrow] (revision) -- (excluidos);
			
		\end{tikzpicture}
	\end{figure}
	
	\begin{table}[ht]
		\centering
		\caption{Bases de datos y términos de búsqueda empleados para la revisión sobre cifrado de extremo a extremo}
		\label{tab:busqueda}
		\vspace{0.5cm}
		\begin{tabularx}{\textwidth}{l X}
			\hline
			\textbf{Base de datos} & \textbf{Términos de búsqueda} \\
			\hline
			Scopus & “end-to-end encryption”, “digital privacy”, “data protection”, “cryptography”, “secure communications” \\
			\hline
			IEEE Xplore & “end-to-end encryption”, “information security”, “secure messaging”, “data confidentiality”, “privacy protection” \\
			\hline
			arXiv & “end-to-end encryption”, “cryptography”, “privacy”, “digital security”, “secure communication protocols” \\
			\hline
		\end{tabularx}
	\end{table}
	
	
		
	\begin{multicols}{2}
		
		\subsection*{\normalsize Análisis de la variabilidad y consistencia de las fuentes }
		
		Al realizar el análisis comparativo de las diferentes fuentes consultadas, se observó que existen tanto puntos en común como diferencias en la manera en que los distintos autores abordan el tema del cifrado de extremo a extremo. Estas diferencias y semejanzas se analizaron tomando en cuenta los distintos contextos en los que se aplica esta tecnología, tales como la mensajería instantánea, el correo electrónico y los servicios en la nube. Además, se consideraron las perspectivas desde las cuales se examina el tema, incluyendo aspectos técnicos, legales y éticos. A pesar de que algunos trabajos muestran enfoques y énfasis distintos, la mayoría coincide en valorar el cifrado E2EE como una herramienta esencial para proteger la información frente a accesos no autorizados. En general, los hallazgos reflejan una tendencia clara hacia el reconocimiento del cifrado de extremo a extremo como un mecanismo clave para la defensa de la privacidad y la seguridad digital, subrayando su importancia creciente en un mundo cada vez más conectado y vulnerable a diversas amenazas. Esta consistencia, a pesar de la diversidad de enfoques, contribuye a fortalecer la confianza en el uso del cifrado como medida de protección indispensable.
	\end{multicols}
		\section*{\normalsize Desarrollo y Discución }
		\begin{multicols}{2}
		El cifrado de extremo a extremo (E2EE) se ha consolidado como una herramienta fundamental para proteger nuestras comunicaciones digitales. Cuando enviamos un mensaje o realizamos una transacción en línea, esta tecnología garantiza que únicamente nosotros y el destinatario podamos acceder al contenido, funcionando como un sistema de seguridad que codifica la información durante todo su recorrido.
		La revisión sistemática de la literatura en Scopus reveló una distribución heterogénea de investigaciones sobre cifrado de extremo a extremo (E2EE) entre 2022-2024. Los estudios analizados se agruparon en tres categorías principales según su enfoque de aplicación:
	\end{multicols}
	\subsection*{\normalsize Distribución por sector de aplicación:}
	\begin{figure}[H]
		\centering
		\begin{tikzpicture}
			\begin{axis}[
				xbar,
				width=8cm,
				height=5cm,
				xlabel={Porcentaje de publicaciones},
				symbolic y coords={
					Servicios financieros y transacciones digitales,
					Sistemas de comunicación y redes sociales,
					Internet de las Cosas y dispositivos conectados
				},
				ytick=data,
				nodes near coords,
				xmin=0,
				xmax=50,
				enlarge y limits=0.2,
				bar width=12pt,
				xtick={0,10,20,30,40,50},
				xlabel style={font=\bfseries},
				yticklabel style={text width=9cm, align=right}
				]
				\addplot coordinates {
					(45,Internet de las Cosas y dispositivos conectados)
					(32,Sistemas de comunicación y redes sociales)
					(23,Servicios financieros y transacciones digitales)
				};
			\end{axis}
		\end{tikzpicture}
	\end{figure}
	\begin{itemize}
		\item Internet de las Cosas y dispositivos conectados
		\item Sistemas de comunicación y redes sociales
		\item Servicios financieros y transacciones digitales
	\end{itemize}
	
	La metodología empleada para la organización de datos siguió el protocolo PRISMA, permitiendo una categorización sistemática que facilitó la identificación de patrones emergentes y brechas en la investigación actual.
	\subsubsection*{Aplicación del Protocolo PRISMA}
	
	La organización y análisis de los datos se realizó siguiendo el protocolo PRISMA, el cual consta de cuatro fases principales:
	
	\begin{itemize}
		\item \textbf{Identificación}: Se realizó una búsqueda sistemática en bases de datos académicas (Scopus, IEEE Xplore y ScienceDirect) utilizando términos relacionados con “cifrado de extremo a extremo”, “seguridad digital” y “comunicaciones seguras”.
		\item \textbf{Cribado}: Se aplicaron criterios de inclusión y exclusión para filtrar los registros relevantes. Se eliminaron duplicados y se evaluaron los títulos y resúmenes.
		\item \textbf{Elegibilidad}: Los artículos seleccionados fueron leídos en su totalidad para determinar su relevancia metodológica y temática.
		\item \textbf{Inclusión}: Se incluyeron en la síntesis final aquellos estudios que cumplían con los requisitos de calidad y pertinencia.
	\end{itemize}
	\subsubsection*{Mapa Mental}
	\begin{figure}[H]
		\centering
		\begin{tikzpicture}[node distance=1.5cm and 1cm]
			
			% Nodo central
			\node[concept] (central) {Cifrado de Extremo a Extremo};
			
			% Rama 1 - Aplicaciones Tecnológicas (arriba a la derecha)
			\node[subconcept, above right=of central] (app1) {IoT y dispositivos inteligentes \\ \textit{(Lin et al., 2024)}};
			\node[subconcept, right=of app1] (app2) {Sistemas de publicación-suscripción};
			\node[subconcept, below=of app1] (app3) {Arquitecturas de re-cifrado proxy};
			
			% Rama 2 - Implicaciones Sociales (abajo a la izquierda)
			\node[subconcept, below left=of central] (soc1) {Privacidad vs. seguridad pública};
			\node[subconcept, left=of soc1] (soc2) {Investigación criminal y forense digital};
			\node[subconcept, below=of soc1] (soc3) {Marcos regulatorios emergentes};
			
			% Rama 3 - Implementación Práctica (arriba a la izquierda)
			\node[subconcept, above left=of central] (imp1) {Sistemas de pago digital};
			\node[subconcept, left=of imp1] (imp2) {Protocolos de autenticación};
			\node[subconcept, below=of imp1] (imp3) {Estándares de cumplimiento normativo};
			
			% Conexiones
			\draw[arrow] (central) -- (app1);
			\draw[arrow] (central) -- (app2);
			\draw[arrow] (central) -- (app3);
			
			\draw[arrow] (central) -- (soc1);
			\draw[arrow] (central) -- (soc2);
			\draw[arrow] (central) -- (soc3);
			
			\draw[arrow] (central) -- (imp1);
			\draw[arrow] (central) -- (imp2);
			\draw[arrow] (central) -- (imp3);
			
		\end{tikzpicture}
		\caption{Mapa mental de enfoques del cifrado de extremo a extremo (E2EE)}
	\end{figure}
	\subsection*{Combinación de los Resultados de Diferentes Artículos}
	\begin{multicols}{2}
	\paragraph{Convergencias Identificadas.} 
	Los estudios analizados coinciden en varios aspectos fundamentales. La investigación de \textit{Lin et al.} (2024) sobre \textit{Internet de las Cosas (IoT)} complementa los hallazgos de estudios previos sobre eficiencia energética en dispositivos con recursos limitados. Paralelamente, los análisis de la Universidad de Johannesburgo sobre el impacto del \textit{cifrado de extremo a extremo (E2EE)} en investigaciones criminales encuentran eco en múltiples publicaciones que abordan el equilibrio entre privacidad y seguridad.
	
	\begin{itemize}
		\item Necesidad de optimización para dispositivos con limitaciones de procesamiento.
		\item Tensión constante entre privacidad individual y necesidades de seguridad colectiva.
		\item Evolución acelerada de marcos regulatorios internacionales.
		\item Incremento en costos de implementación versus beneficios de seguridad.
	\end{itemize}
	
	\paragraph{Divergencias y Enfoques Complementarios.} 
	Mientras algunos estudios priorizan la eficiencia técnica del cifrado, otros se enfocan en las implicaciones éticas y legales. Esta diversidad de perspectivas enriquece el panorama investigativo, aunque también revela la necesidad de una mayor integración interdisciplinaria.
	\end{multicols}
	\subsection*{Argumentación Crítica de los Resultados}
	\begin{multicols}{2}
	
	\paragraph{Análisis de Diseños Metodológicos.}
	
	\textbf{Fortalezas identificadas:} La investigación de \textit{Lin et al.} (2024) presenta un diseño experimental robusto con implementación práctica en entornos controlados.\\ Su enfoque en \textit{re-cifrado proxy condicional} aporta una solución innovadora que ha sido validada empíricamente, demostrando reducciones significativas en consumo energético.
	
	\textbf{Limitaciones metodológicas detectadas:} Sin embargo, varios estudios presentan limitaciones notables. Los análisis sobre el impacto del E2EE en la investigación criminal se basan principalmente en datos retrospectivos, lo cual puede introducir sesgos de selección. Además, muchas investigaciones sobre sistemas de pago se realizan en entornos simulados que no replican completamente las condiciones operativas reales.
	
	\paragraph{Identificación de Sesgos.}
		\textbf{\\Sesgo de publicación:} La literatura muestra una tendencia hacia resultados positivos, con escasa documentación de implementaciones fallidas o parcialmente exitosas del E2EE.\\
		\textbf{Sesgo geográfico:} La mayoría de estudios provienen de instituciones en países desarrollados, limitando la generalización de los resultados a contextos con diferentes marcos regulatorios y niveles de desarrollo tecnológico.\\
		\textbf{Sesgo temporal:} La rápida evolución tecnológica puede hacer que algunos hallazgos pierdan relevancia en períodos relativamente cortos.

	
	\paragraph{Limitaciones Sistémicas.}
	\begin{itemize}
		\item \textbf{Complejidad interdisciplinaria:} Los estudios tienden a especializarse en aspectos técnicos o sociales, pero rara vez integran ambas perspectivas de manera comprehensiva.
		\item \textbf{Escalabilidad:} Muchas propuestas muestran viabilidad en entornos controlados, pero presentan incertidumbres sobre su implementación a gran escala.
		\item \textbf{Consideraciones económicas:} Existe una brecha entre la factibilidad técnica y la viabilidad económica, especialmente para organizaciones de menor tamaño.
	\end{itemize}
	\end{multicols}
	\subsection*{Conclusiones Extraídas y Su Validez}
		\begin{multicols}{2}
	
	\paragraph{Conclusiones sólidas respaldadas por evidencia.}
	\begin{itemize}
		\item El cifrado de extremo a extremo (E2EE) es técnicamente viable para dispositivos \textit{IoT}, siempre que se apliquen optimizaciones apropiadas.
		\item Existe un impacto medible en los procedimientos de investigación criminal, evidenciado por estudios de caso y revisiones institucionales.
		\item Los costos de implementación del E2EE varían significativamente según el sector de aplicación, influidos por factores como escalabilidad, regulación y capacidad tecnológica.
	\end{itemize}
	
	\paragraph{Conclusiones que requieren mayor investigación.}
	\begin{itemize}
		\item El equilibrio óptimo entre privacidad individual y seguridad pública sigue sin resolverse, debido a tensiones éticas, legales y técnicas.
		\item La interoperabilidad entre diferentes sistemas y protocolos de E2EE presenta desafíos aún no completamente abordados por la literatura.
		\item Los efectos a largo plazo derivados de la adopción masiva del E2EE requieren estudios longitudinales más extensos y comparativos entre regiones.
	\end{itemize}
		\end{multicols}
	\section*{Referencias}
	%=====================================================================
	% 				PARA REFERENCIAS
	%=====================================================================
	\begin{multicols}{2}
		\printbibliography
	\end{multicols}
	

	
\end{document}
